66c66
< V~.H{\'e}nault-Brunet$^{6}$,\\
---
> V.~H{\'e}nault-Brunet$^{6}$,
96,98c96,98
< We find an average metallicity for NGC\,2100 of [Z]~=~$-$0.39\,$\pm$\,0.22\,dex, in good agreement with estimates from the literature for the LMC.
< Comparing our results in NGC\,2100 with those for a Galactic cluster (at Solar-like metallicity) with a similar mass and age we find no significant change in the location of RSGs in the H--R diagram.
< We combine the observed KMOS spectra to form a simulated integrated cluster spectrum and show that this is consistent with the average properties of the cluster.
---
> We find an average metallicity for NGC\,2100 of [Z]~=~$-$0.39\,$\pm$\,0.20\,dex, in good agreement with estimates from the literature for the LMC.
> Comparing our results in NGC\,2100 with those for a Galactic cluster (at Solar-like metallicity) with a similar mass and age we find no significant difference in the location of RSGs in the Hertzsprung--Russell diagram.
> We combine the observed KMOS spectra to form a simulated integrated-light cluster spectrum and show that this is consistent with the average properties of the cluster.
101c101
< The data is consistent with a flat velocity dispersion profile, and an upper limit of 3.9\,\kms,at the 95\% confidence level, on the velocity dispersion of the cluster.
---
> The data are consistent with a flat velocity dispersion profile, and with an upper limit of 3.9\,\kms, at the 95\% confidence level, for the velocity dispersion of the cluster.
104c104
< $M_{dyn}$~=~$15.2\times10^{4}M_{\odot}$ assuming virial equilibrium.
---
> $M_{dyn}$~\le~$15.2\times10^{4}M_{\odot}$ assuming virial equilibrium.
116,117c116,117
< In addition to being the birthplace of most of the massive stars in the Local Universe~\citep[$>200\,$M$_{\odot}$ stars in R136;][]{2010MNRAS.408..731C}, owing to the density of stars, YMCs are supposedly the birthplace of some of the rich stellar exotic
< (e.g. blue stragglers, X-ray binaries and radio pulsars) found in the old population of globular clusters~\citep[GCs;]{2010ARA&A..48..431P}.
---
> In addition to being the birthplace of most of the massive stars in the Local Universe~\citep[$>200\,$M$_{\odot}$ stars in R136;][]{2010MNRAS.408..731C}, owing to the density of stars, YMCs are thought to be the birthplace of some of the rich stellar exotica
> (e.g. blue stragglers, X-ray binaries and radio pulsars) found in the old population of globular clusters~\citep[GCs;][]{2010ARA&A..48..431P}.
119c119
< \footnotetext{A YMC is defined as $<100\,$Myr and $>10^{4}\,$M$_{\odot}$~\citep{2010ARA&A..48..431P}.}
---
> \footnotetext{A YMC is defined as having an age of $<100\,$Myr and a stellar mass of $>10^{4}\,$M$_{\odot}$~\citep{2010ARA&A..48..431P}.}
126c126
< As most stellar systems dissolve shortly after formation~\citep{2003ARA&A..41...57L}, determining how long bound systems can remain so is an important question to answer.
---
> As most stellar systems are thought to dissolve shortly after formation~\citep{2003ARA&A..41...57L}, determining how long bound systems can remain so is an important question to answer.
128c128
< In addition, the study of YMCs in different environments can help bridge the gap between the understanding of star formation in the Solar-neighbourhood and that in the high--redshift Universe.
---
> In addition, the study of YMCs in different environments can help bridge the gap between the understanding of star formation in the Solar neighbourhood and that in the high-redshift Universe.
133c133
< In star-forming galaxies, RSGs are the most luminous near-IR sources, therefore, by exploiting these wavelengths they can be observed out to large distances.
---
> In star-forming galaxies, RSGs are the most luminous near-IR sources, therefore, they can be observed out to large distances at these wavelengths.
169c169
< The observations consisted of $8\times10$\,s exposures (seeing conditions $\sim$1\farcs0) taken with the YJ grating with sky offset exposures (S) interleaved between the object exposures (O) in an O,~S,~O observing pattern.
---
> The observations consisted of $8\times10$\,s exposures (seeing conditions $\sim$1\farcs0) taken with the $YJ$ grating with sky offset exposures (S) interleaved between the object exposures (O) in an O,~S,~O observing pattern.
199c199
<         Observed properties of VLT-KMOS targets in NGC\,2100\label{tb:obs-params}
---
>         Observed properties of VLT-KMOS targets in NGC\,2100.\label{tb:obs-params}
273c273
< Obvious outliers (with $\delta$RVs of tens of km\,s$^{-1}$) were excluded in calculating the mean estimates; such outliers arise occasionaly from spurious peaks in the cross-correlation functions from noise/systematics in the spectra.
---
> Obvious outliers (with $\delta$RVs of tens of km\,s$^{-1}$) were excluded in calculating the mean estimates; such outliers arise occasionally from spurious peaks in the cross-correlation functions from noise/systematics in the spectra.
346c346
<         Literature stellar radial velocity measurements within NGC\,2100\label{tb:rvs}
---
>         Literature stellar radial-velocity measurements within NGC\,2100.\label{tb:rvs}
542,545c542,545
< (black and red lines respectively).
< Top panel shows the simulated integrated-light cluster spectrum.
< Bottom panel shows spectra for the individual RSGs.
< The lines used for the analysis from left-to-right by species are
---
> (black and red lines, respectively).
> The upper panel shows the simulated integrated-light cluster spectrum;
> the lower panel shows spectra for the individual RSGs.
> The lines used for the analysis, from left-to-right by species, are
569c569
< Model grid used for the spectroscopic analysis\label{tb:mod_range}
---
> Model grid used for the spectroscopic analysis.\label{tb:mod_range}
590c590
< Physical parameters determined for the KMOS targets in NGC\,2100
---
> Physical parameters determined for the KMOS targets in NGC\,2100.
632c632
< NGC\,2100 average & & 3.9\,$\pm$\,0.8 & -0.39\,$\pm$\,0.20 &  0.29\,$\pm$\,0.18 & 3890\,$\pm$\,85 &\\
---
> NGC\,2100 average & & 3.9\,$\pm$\,0.8$\phantom{0}$ & $-$0.39\,$\pm$\,0.20 &  0.29\,$\pm$\,0.18 & 3890\,$\pm$\,85 &\\
664c664
<  \caption{H--R diagram for 12 RSGs in NGC\,2100 (black points).
---
>  \caption{H--R diagram for 14 RSGs in NGC\,2100 (black points).
675c675
< as NGC\,2100, and a comparison between the stellar components of these two clusters using a consistent analysis technique is useful to highlight differences in stellar evolution within clusters at different metallicities.
---
> as NGC\,2100, and a comparison between the stellar components of these two clusters using a consistent analysis technique is useful to highlight differences in stellar evolution within clusters at this range of metallicities.
714c714
< a surface gravity of 0.64\,$\pm$\,0.19\,dex and a microturbulent velocity of $4.6\pm0.2\,$\kms~which agree well with the average of the individual RSG parameters.
---
> a surface gravity of 0.64\,$\pm$\,0.19\,dex and a microturbulent velocity of $4.6\pm0.2\,$\kms~which agree well with the averages of the individual RSG parameters.
790c790,791
< In addition to estimating the dynamical properties of NGC\,2100, we have also reliably estimated stellar parameters for 13 RSGs in NGC\,2100 using the new $J$-band analysis technique~\citep{2010MNRAS.407.1203D}.
---
> In addition to estimating the dynamical properties of NGC\,2100, we have also %reliably
> estimated stellar parameters for 14 RSGs in NGC\,2100 using the new $J$-band analysis technique~\citep{2010MNRAS.407.1203D}.
794c795
< Using stellar parameters estimated from RSGs using the same analysis technique as that in this study, we demonstrate that there exists no significant difference in the appearance of the H--R diagram of YMCs between Solar- and LMC-like metallicities.
---
> Using stellar parameters estimated from RSGs using the same analysis technique as in this study, we demonstrate that there exists no significant difference in the appearance of the H--R diagram of YMCs between Solar- and LMC-like metallicities.
